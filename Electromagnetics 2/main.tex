%% ------------------------------------------------------------
%% TITLE:     Electromagnetics - Formula Summary
%% AUTHOR:    BINGHUAN W LI (Dept. Chemical Eng/Bio Eng, Imperial)
%%            PETER Y XIE	(Dept. Mech Eng, Stanford)
%% COMPILED:  XeLaTeX with TeX Live version 2023
%% LICENSE:   This work is licensed under a Creative Commons Attribution-NonCommercial 4.0 International License.
%% ------------------------------------------------------------

% Version History:
% v1.0  - 2021-05-13 - Initial draft
% v1.1  - 2023-09-28 - Reformatted the document

\documentclass[a4paper, 11pt]{article}
\usepackage[margin=1cm]{geometry}
% \usepackage{amsmath, amsfonts, mathrsfs}
\usepackage{newtxmath}
\usepackage{fontspec}
    \setmainfont{Times New Roman}

\usepackage{hyperref}
\usepackage{float}
\usepackage{booktabs}
\usepackage{enumitem}

\usepackage{multirow, multicol}
\setlength{\columnseprule}{1pt}

\usepackage{fancyhdr}
\pagestyle{empty}
\fancyhf{}

\hypersetup{pdfauthor={Li, Binghuan}}

% disable paragraph indentation
\setlength\parindent{0pt}

% define hat{n}, hat{x}, hat{y}, hat{z}
\DeclareMathOperator{\hn}{\hat{\boldsymbol{n}}}
\DeclareMathOperator{\hx}{\hat{\boldsymbol{x}}}
\DeclareMathOperator{\hy}{\hat{\boldsymbol{y}}}
\DeclareMathOperator{\hz}{\hat{\boldsymbol{z}}}

\begin{document}
\underline{\Large{\textbf{Electromagnetics - Formula Summary}}} \vspace{.5cm}

\begin{enumerate}
    \item Coulomb's Law: $\vec{F} = \dfrac{q_{1} \ q_{2}}{4 \ \pi \ \varepsilon_{r} \ \varepsilon_{0} \ r^{2}}$

    \item Electric Field: $\vec{E} = \dfrac{\vec{F}}{q}$
    \begin{table}[H]
        \centering
        \begin{tabular}{cccc}
        \toprule
        Field of a point charge &
        Field inside a capacitor &
        Superposition &
        Electric Flux \\
        \midrule
        $\displaystyle E = \frac{kQ}{r^{2}}$ &
        $\displaystyle E = \frac{V}{d}$ &
        $\displaystyle \vec{E}_{net} = \sum_{i=1}^{N}\vec{E}_{i}$ &
        $\displaystyle \Phi_{E} = \int \vec{E} \cdot \mathrm{d}\vec{A}$ \\
        \bottomrule
        \end{tabular}
    \end{table}
    

    \item Electrical Potential: $V = \dfrac{U}{q}$, \quad $\displaystyle \Delta V = V_{f}-V_{i} = -\int_{i}^{f} \vec{E}\ \mathrm{d} \vec{l}$


    \item Gauss's Law:
    \[
        \oint \vec{E} \cdot \mathrm{d}\vec{A} = \frac{Q_{\rm enc}}{\varepsilon}, 
        \quad \quad \quad 
        \nabla \cdot \vec{E} = \frac{\rho}{\varepsilon}
    \]
    \[
        \oint \vec{B} \cdot \mathrm{d}\vec{A} = 0, 
        \quad \quad \quad 
        \nabla \cdot \vec{B} = 0
    \]
    
    \item Ampere's Law:
    \[
        \oint \vec{B} \cdot \mathrm{d}\vec{l} = \mu \int \left( \vec{J}+\varepsilon \frac{\partial \vec{E}}{\partial t} \right) \cdot \mathrm{d}\vec{A},
        \quad \quad
        \nabla \times \vec{B} = \mu \left( \vec{J}+\varepsilon \frac{\partial \vec{E}}{\partial t} \right)
    \]
    
    \item Faraday's Law:
    \[
        \oint \vec{E} \cdot \mathrm{d}\vec{l} = -\frac{\mathrm{d}}{\mathrm{d}t} \int\vec{B} \cdot \mathrm{d}\vec{A},
        \quad \quad
        \nabla 
        \times \vec{E} = - \frac{\partial \vec{B}}{\partial t}
    \]

    \item Magnetic Field:
    \begin{itemize}
        \item A straight line wire: $B = \dfrac{\mu_{0} \ I}{2 \ \pi \ r}$
        \item Flux: $\Phi_{B} = \int\vec{B} \cdot \mathrm{d}\vec{A}$
    \end{itemize}

    \item Biot-Savart Law: $\displaystyle \vec{B} = \oint\frac{\mu \ I \ \mathrm{d}\vec{l}\times \hat{r}}{4 \pi r^{2}}$

    \item Ohm's Law: $\vec{J} = \sigma \vec{E}$

    \item Inductance: $L = \dfrac{\Phi_{T}}{I} = \dfrac{\mu_{0}N^{2}A}{l}$, \quad \quad $\varepsilon = -L\dfrac{\mathrm{d}I}{\mathrm{d}t}$

    \item Capacitance: $\displaystyle C = \frac{Q}{V}$

    \item Poynting Vector: $\displaystyle \vec{S} = \vec{E}\times \vec{H}\quad \text{where} \ \vec{H}=\frac{\vec{B}}{\mu_{0}}$

    \item Total Energy Density: $E_{d}=\dfrac{1}{2}E^{2}\varepsilon+\frac{1}{2}\mu H^{2}$
    
    \item Electromagnetics Waves:
    \begin{itemize}[leftmargin=*]
	\item Speed of wave: $v=\dfrac{1}{\sqrt{\mu\varepsilon}}$
        
	\item Refractive index: $ n=\dfrac{c} {v}=\sqrt{\mu_{r}\varepsilon_{r}}$

	\item Snell's Law: $n_{1} \sin \theta_{1} = n_{2} \sin \theta_{2}, \quad \sin\theta_{c} = \dfrac{n_{2}}{n_{1}}$

        \item Boundary continuity 
        \begin{table}[H] 
            \centering
		\begin{tabular}{ccc}
		\toprule
            & dielectric-dielectric & dielectric-PEC ($\vec{E}_{2}=0$)\\[.2em]
            \midrule
            $||$ & $E_{t,1}=E_{t,2}$,  $H_{t,1}=H_{t,2}$ & $E_{t,1}=0$ \\[.4em]
            
            \multirow{2}{*}{$\perp$} & $\varepsilon_{1}\hn\cdot\vec{E}_{1}-\varepsilon_{2}\hn\cdot\vec{E}_{2}=\rho_{s}$  & $\varepsilon_{1} \hn\cdot\vec{E}_{1}=\rho_{s}$ \\
            &                  $\hn\cdot\vec{H}_{1}=\hn\cdot\vec{H}_{2}$ & $\hn\cdot\vec{H}_{1}=0$, \ $\hn\times \vec{H}_{1}=\vec{J}_{s}$ \\ 
            \bottomrule
		\end{tabular}
        \end{table}
  
	\item Reflection and Transmission Coefficients
        \[
            \Gamma=\frac{\eta_{2}-\eta_{1}}{\eta_{2}+\eta_{1}}, 
            \quad \quad 
            T=\frac{2 \ \eta_{2}}{\eta_{2}+\eta_{1}}
            \quad \quad 
            \text{where} \ \eta=\frac{\lvert \vec{E} \rvert}{\lvert \vec{H} \rvert}=\sqrt{\frac{\mu_{0}}{\varepsilon}} \quad \Rightarrow \quad T=1+\Gamma 
        \]
        
	\item Parallel polarization:
	\begin{itemize}[leftmargin=*]
		\item Incident wave 
            \begin{align*}
                \vec{E}_{i}
                = E_{0} \  e&^{{j(\omega t + k_{1}(x\sin\theta_{i} + y \cos \theta_{i}))}} (-\cos \theta_{i} \hx + \sin \theta_{i} \hy)
            \end{align*}
		\[
                \vec{H}_{i} = 
                \frac{E_{0}}{\eta_{1}} \ e^{j(\omega t+k_{1}(x\sin\theta_{i}+y\cos\theta_{i}))}(-\hz)
            \]
            
		\item Transmitted wave 
            \begin{align*}
                \vec{E}_{t} = TE_{0} \ e& ^{j(\omega t+k_{2}( x \sin \theta_{t} + y \cos \theta_{t}))} (-\cos \theta_{t} \hx + \sin \theta_{t} \hy) 
            \end{align*}
		\[
                \vec{H}_{t} = 
                \frac{TE_{0}}{\eta_{2}} \ e^{j(\omega t+k_{2}(x\sin\theta_{t}+y\cos\theta_{t}))}(-\hz)
            \]
            
		\item Reflected wave 
            \begin{align*}
                \vec{E}_{r} = \Gamma E_{0} \ e & ^{j(\omega t+k_{1}(x\sin\theta_{r}-y\cos\theta_{r}))} (-\cos\theta_{r}\hx-\sin\theta_{r}\hy)    
            \end{align*}
		\[
                \vec{H}_{r}=\frac{\Gamma E_{0}}{\eta_{1}}e^{j(\omega t+k_{1}(x\sin\theta_{r}-y\cos\theta_{r}))}\hz
            \]
	\end{itemize}
 
	\item Fresnel Equations 
        \[
            T_{\parallel}= \frac{2\eta_{2}\cos\theta_{i}}{\eta_{2}\cos\theta_{t}+\eta_{1}\cos\theta_{i}},
            \quad \quad
            \Gamma_{\parallel} = \frac{\eta_{2}\cos\theta_{t}-\eta_{1}\cos\theta_{i}}{\eta_{2}\cos\theta_{t}+\eta_{1}\cos\theta_{i}}
        \]
        \[
            T_{\perp}= \frac{2\eta_{2}\cos\theta_{i}}{\eta_{2}\cos\theta_{i}+\eta_{1}\cos\theta_{t}},
            \quad \quad
            \Gamma_{\perp} = \frac{\eta_{2}\cos\theta_{i}-\eta_{1}\cos\theta_{t}}{\eta_{2}\cos\theta_{i}+\eta_{1}\cos\theta_{t}}
        \]
        
	\item Brewster angle: $\displaystyle \sin\theta_{i}=\frac{1}{\sqrt{1+\varepsilon_{1}/\varepsilon_{2}}}$
\end{itemize}
\end{enumerate}

\section*{Constants}
\begin{itemize}
    \item Charge on electron: $e = 1.60 \times 10^{-19} \ \ \mathrm{C}$
    
    \item Permittivity of free space: $\varepsilon_{0} = 8.85 \times 10^{-12} \ \ \mathrm{C^{2}/Nm^{2}}$
    
    \item Permeability of free space: $\mu_{0}  = 4\pi \times 10^{-7} \ \ \mathrm{Tm/A}$
    
    \item $k = \frac{1}{4\pi \varepsilon_{0}} = 8.99\times 10^{9} \ \ \mathrm{Nm^{2}/C^{2}}$
    
    \item Light of speed: $c = \frac{1}{\sqrt{\mu_{0}\varepsilon_{0}}} = 3.0\times 10^{8} \ \ \mathrm{m/s}$
\end{itemize}

% end of document
\vspace*{\fill}
\framebox{Scripted by B Li \& P Xie. \ Last update: \today}
\end{document}