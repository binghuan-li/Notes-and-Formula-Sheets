\section*{Acknowledgement}
I would like to express my deepest appreciation to \emph{Prof. Dario Farina} for his help in reviewing these notes with the highest profession. \\

I would like to extent my sincere thanks to \emph{Rea Tresa}, \emph{Ben Ford}, and \emph{Elisa Soliani} for their insightful comments. Their work significantly optimized the structure. \\

Special thanks to \emph{Haroon Chughtai}. His notes greatly enlightened me when preparing this new set.

\section*{Update Notes}
\subsection*{10\textsuperscript{th} Dec, 2022}
\begin{enumerate}
    \item Fixed some typos found in the notes.
    \item Improved the layout of all derivations and examples.
    \item Removed the redundant descriptions in \textit{Section 6.3}.
\end{enumerate}
\subsection*{13\textsuperscript{th} Jan, 2021}
\begin{enumerate}
    \item  \emph{Section 3.5.3}: Corrected typos: missing comma in $\langle f_{1}(t), f_{2}(t) \rangle$.
    \item \emph{Section 5.3.8}: Corrected math typos in the derivation: missing $d\tau$ in line 1, $d\tau \to d\alpha$ in line 3.
    \item \emph{Section 5.3.8}: Corrected typos: missing $\omega_{0}$ in $\cos(\omega_{0} t)$.
    \item \emph{Section 5.6}: Added the formula for finding the phase angle. Added a plot for $sign(x)$.
\end{enumerate}
\subsection*{27\textsuperscript{th} Dec, 2020}
\begin{enumerate}
    \item \emph{Section 5.3.2}: Added the derivation of time shifting property.
    \item \emph{Section 5.5}: Corrected math typos, rearranged the example and improved the readability.
    \item \emph{Section 5.5.1}: Added a description of LTI systems.
    \item \emph{Section 5.6}: Refined the description of magnitude and phase spectra. An example is adopted from the textbook.
\end{enumerate}


\section*{Source File}
\faGithub \ click to access the GitHub repository for \href{https://github.com/binghuan-li/Notes-and-Formula-Sheets}{this document}.