\section*{Preface, Rationale, \& Acknowledgements}
This document was initially constructed and released in November 2020 as a collection of my class notes for the first element \textit{Signals and Systems} in the module \textit{BIOE50011-Signals and Control} (same syllabus applied to the module \textit{BIOE50005 - Mathematics and Engineering 2}) at the Department of Bioengineering, Imperial College London. Multiple insightful suggestions and requirements came in the subsequent years regarding the update and maintenance of this document, and surprisingly, I am still able to pick up those contents and revisit them as if they are from the lectures I attended yesterday. \\

For anyone who is using this document - I really wish you can go well with the signals, rather than just learn for the somehow tricky progress test or the final exam, but really appreciate the magic and use it to facilitate your own research. Believe it or not, you are not likely to regret not learning signals well before permanently quitting them and moving to cardiovascular fluid research (Yea, that's my research). \\

I would like to express my deepest appreciation to \emph{Prof. Dario Farina} for his help in reviewing these notes with the highest profession. I would also like to extend my sincere thanks to \emph{Rea Tresa}, \emph{Ben Ford}, and \emph{Elisa Soliani} for their comments. Their work significantly optimized the structure. Finally, my special thanks go to \emph{Haroon Chughtai}. His notes greatly enlightened me when preparing this new set. \\

\faGithub \ The \LaTeX \ files are now accessible on my \href{https://github.com/binghuan-li/Notes-and-Formula-Sheets}{GitHub repository}. I hope it helps. 

\begin{flushright}
March, 2023
\end{flushright}

% \section*{Errata}
% \subsection*{13\textsuperscript{th} Feb, 2023}
% Fixed some typos in the text, and updated figures. More examples are now integrated into the notes.
% \subsection*{10\textsuperscript{th} Dec, 2022}
% \begin{enumerate}[itemsep=.5mm]
%     \item Fixed some typos found in the notes.
%     \item Improved the layout of all derivations and examples.
%     \item Removed the redundant descriptions in \textit{Section 6.3}.
% \end{enumerate}
% \subsection*{13\textsuperscript{th} Jan, 2021}
% \begin{enumerate}[itemsep=.5mm]
%     \item  \emph{Section 3.5.3}: Corrected typos: missing comma in $\langle f_{1}(t), f_{2}(t) \rangle$.
%     \item \emph{Section 5.3.8}: Corrected math typos in the derivation: missing $d\tau$ in line 1, $d\tau \to d\alpha$ in line 3.
%     \item \emph{Section 5.3.8}: Corrected typos: missing $\omega_{0}$ in $\cos(\omega_{0} t)$.
%     \item \emph{Section 5.6}: Added the formula for finding the phase angle. Added a plot for $sign(x)$.
% \end{enumerate}
% \subsection*{27\textsuperscript{th} Dec, 2020}
% \begin{enumerate}[itemsep=.5mm]
%     \item \emph{Section 5.3.2}: Added the derivation of time shifting property.
%     \item \emph{Section 5.5}: Corrected math typos, rearranged the example and improved the readability.
%     \item \emph{Section 5.5.1}: Added a description of LTI systems.
%     \item \emph{Section 5.6}: Refined the description of magnitude and phase spectra. An example is adopted from the textbook.
% \end{enumerate}